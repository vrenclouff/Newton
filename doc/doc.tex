\documentclass{style}
\usepackage[utf8]{inputenc} 
\usepackage[czech]{babel}
\usepackage{ae}
\usepackage{url}
\usepackage{fancyhdr}
\usepackage{listings}
\usepackage{mathtools}
\usepackage{array,tabularx}
\usepackage{graphicx}
\usepackage{float}

\author{Lukáš Černý, Štěpán Baratta}
\title{Překladač navrženého jazyka Newton}
\titlet{}
\titlett{}
\university{Západočeská univerzita v Plzni}
\faculty{Fakulta aplikovaných věd}
\department{Katedra informatiky a výpočetní techniky}
\subject{Semestrální práce z KIV/FJP}
\town{Plzeň}

\begin{document}

\pagestyle{fancy}
\renewcommand{\chaptermark}[1]{\markboth{\textit{#1}}{}}
\renewcommand{\sectionmark}[1]{\markright{\textit{#1}}{}}
\cfoot{\thepage}
\lhead{\leftmark}
\rhead{\rightmark}
\maketitle

\tableofcontents
\pagestyle{fancy}
\renewcommand{\chaptermark}[1]{\markboth{\textit{#1}}{}}
\renewcommand{\sectionmark}[1]{\markright{\textit{#1}}{}}
\cfoot{\thepage}
\lhead{\leftmark}
\rhead{\rightmark}
\parskip 1em

\chapter{Členové týmu}

\begin{table}[h]
\centering
\begin{tabular}{c|c|c}
Jméno & Studijní číslo & Email \\
\hline
Lukáš Černý & xxxx & xxxx \\
Štěpán Baratta & A17N0061P & BarattaStepan@gmail.com \\
\end{tabular}
\end{table}

Odkaz na Github: xxx

\chapter{Zadání}
Cílem této semestrální práce je vytvoření překladače vlastního nebo již existujícího jazyka. Dále je nutné zvolit, pro jakou architekturu bude jazyk překládán.

Mezi základní podmínky, které jazyk musí splňovat, jsou následující konstrukce:

\begin{itemize}
\item definice celočíselných proměnných \\
\item definice celočíselných konstant \\
\item přiřazení \\
\item základní aritmetiku a logiku \verb|(+, -, *, /, AND, OR, negace a závorky, relační operátory)| \\
\item cyklus (libovolný) \\
\item jednoduchou podmínku (if bez else) \\
\item definice podprogramu (procedura, funkce, metoda) a jeho volání \\
\end{itemize}

\chapter{Popis řešení}
Výstup programu bude posloupnost instrukcí vstupního programu. Instrukce jsou generované pro architekturu jazyka PL/0. 

Seznam všech instrukcí je následující:

\begin{itemize}
\item LIT --  0,A    ulož konstantu A do zásobníku \\
\item OPR -- 0,A    proveď instrukci A \\
\begin{itemize}
\item 1    unární minus \\
\item 2    + \\
\item 3    - \\
\item 4    * \\
\item 5    div - celočíselné dělení (znak /) \\
\item 6    mod - dělení modulo (znak %) \\
\item 7    odd - test, zda je číslo liché \\
\item 8    test rovnosti (znak =) \\
\item 9    test nerovnosti (znaky <>) \\
\item 10    < \\
\item 11    >= \\
\item 12    > \\
\item 13    <= \\
\end{itemize}

\item LOD L,A --    ulož hodnotu proměnné z adr. L,A na vrchol zásobníku \\
\item STO L,A --    zapiš do proměnné z adr. L,A hodnotu z vrcholu zásobníku \\
\item CAL L,A  --  volej proceduru A z úrovně L \\
\item INT 0,A   -- zvyš obsah top-registru zásobníku o hodnotu A \\
\item JMP 0,A   -- proveď skok na adresu A \\
\item JMC 0,A   -- proveď skok na adresu A, je-li hodnota na vrcholu zásobníku 0 \\
\item RET 0,0  --  návrat z procedury (return) \\

\end{itemize}

\section{Vytvoření AST (abstract syntax tree)}
Program je implementován v programovacím jazyce Java. Pro vytvoření lexikálního analyzátoru a parseru je použita knihovna ANTLR. Obecný postup při vytváření překladače pomocí ANTLR obsahuje vytvoření gramatiky, která bude popisovat zvolený programovací jazyk. Obsahuje tedy lexikální a parsovací pravidla. Z nich ANTLR vytvoří lexikální analyzátor, který čte vstupní text a rozdělí ho na jednotlivé části (tokeny). Dále jsou tyto tokeny předány parseru, který vytváří abstraktní syntaktický strom a interpretuje kód.

\section{Procházení stromu}
Abstraktní syntaktický strom, který generuje ANTLR je procházen pomocí návrhového vzoru visitor, kdy jsou postupně navštěvovány jednotlivé uzly stromu. Pro každý uzel (pravidlo) je vytvořena metoda, která vykoná potřebný kód pro dané pravidlo a vygeneruje instrukce.

\chapter{Implementace}

\section{Implementovaná rozšířená funkčnost}
\subsection{Jednoduchá rozšíření}
\begin{itemize}
\item cykly \\
\begin{itemize}
\item for \\
\item while \\
\item do while \\
\item repeat until \\
\end{itemize}
\item datový typ boolean \\
\item rozvětvená podmínka \verb|switch| \\
\item násobné přiřazení \\
\item ternární operátor \\
\item paralelní přiřazení \\
\end{itemize}

\subsection{Složitější rozšíření}
\begin{itemize}
\item parametry předávané hodnotou \\
\item návratová hodnota podprogramu \\
\end{itemize}

\section{Konstrukce programu}
\subsection{for}
\subsubsection{gramatika}
\begin{lstlisting}
BeginFor Identifier Assign factor Colon factor (Colon Int)? 
Do 
	statement* 
EndFor;
\end{lstlisting}

\subsubsection{použití}
\begin{lstlisting}
for a = 1 : 10 : 2 do
	b = b + 1;
endfor
\end{lstlisting}

\subsection{while}
\subsubsection{gramatika}
\begin{lstlisting}
BeginWhile expression Do statement* EndWhile;
\end{lstlisting}

\subsubsection{použití}
\begin{lstlisting}
while a < 10 do
	b = b + 1;
endwhile
\end{lstlisting}

\subsection{do while}
\subsubsection{gramatika}
\begin{lstlisting}
Do statement* BeginWhile expression;
\end{lstlisting}

\subsubsection{použití}
\begin{lstlisting}
do 
	b = b + 1;
while a > 1;
\end{lstlisting}

\subsection{repeat until}
\subsubsection{gramatika}
\begin{lstlisting}
Repeat statement* Until expression;
\end{lstlisting}

\subsubsection{použití}
\begin{lstlisting}
repeat
	b = b + 1;
until a < 1;
\end{lstlisting}

\subsection{switch}
\subsubsection{gramatika}
\begin{lstlisting}
BeginSwitch simpleExpression Of 
	caseStatement+ 
DefaultSwitch Colon statement EndSwitch;
\end{lstlisting}

\subsubsection{použití}
\begin{lstlisting}
switch param of
    0: result = false;
    1: result = true;
    default: result = false;
endswitch
\end{lstlisting}

\subsection{ternární operátor}
\subsubsection{gramatika}
\begin{lstlisting}
expression Ques expression Colon expression;
\end{lstlisting}

\subsubsection{použití}
\begin{lstlisting}
bool a;
. . .
a = b < 3 ? true : false;
\end{lstlisting}

\subsection{paralelní přiřazení}
\subsubsection{gramatika}
\begin{lstlisting}
CurlyBracketLeft Identifier (',' Identifier)* CurlyBracketRight 
Assign CurlyBracketLeft simpleFactor (',' simpleFactor)* 
CurlyBracketRight Semi;
\end{lstlisting}

\subsubsection{použití}
\begin{lstlisting}
{a, b, c} = {1, 2, 5};
\end{lstlisting}

\section{Vzorové programy}
\begin{minipage}{\linewidth}
\begin{lstlisting}[frame=single]  % Start your code-block

constant:
variable:
    int a;
    int i;
    int j;

    fnc int pocitej(int d)
        for i = 1 : 5 do
            for j = 1 : 5 do
                d = d + 1;
            endfor
        endfor

        return d;
    end

main()
    a = pocitej(0);
end
\end{lstlisting}
\end{minipage}

\subsubsection{Výstup}

\begin{lstlisting}  % Start your code-block
0	JMP	0	1
1	INT	0	7
2	LIT	0	0
3	STO	0	4
4	LIT	0	0
5	STO	0	5
6	LIT	0	0
7	STO	0	6
8	JMP	0	39
9	INT	0	3
10	LIT	0	1
11	STO	1	5
12	LOD	1	5
13	LIT	0	5
14	OPR	0	13
15	JMC	0	36
16	LIT	0	1
17	STO	1	6
18	LOD	1	6
19	LIT	0	5
20	OPR	0	13
21	JMC	0	31
22	LOD	0	-1
23	LIT	0	1
24	OPR	0	2
25	STO	0	-1
26	LOD	1	6
27	LIT	0	1
28	OPR	0	2
29	STO	1	6
30	JMP	0	18
31	LOD	1	5
32	LIT	0	1
33	OPR	0	2
34	STO	1	5
35	JMP	0	12
36	LOD	0	-1
37	STO	1	3
38	RET	0	0
39	LIT	0	0
40	CAL	0	9
41	INT	0	-1
42	LOD	0	3
43	STO	0	4
44	RET	0	0
\end{lstlisting}

\begin{minipage}{\linewidth}
\begin{lstlisting}[frame=single]  % Start your code-block

constant:
variable:
    bool a;
    bool result;

    fnc bool pocitej(int d)
        switch d of
            0: result = false;
            5: result = true;
            default: result = false;
        endswitch

        return result;
    end

main()
    a = pocitej(5);
end
\end{lstlisting}
\end{minipage}

\subsubsection{Výstup}

\begin{lstlisting}  % Start your code-block
0	JMP	0	1
1	INT	0	6
2	LIT	0	0
3	STO	0	4
4	LIT	0	0
5	STO	0	5
6	JMP	0	29
7	INT	0	3
8	LOD	0	-1
9	STO	1	3
10	LOD	1	3
11	LIT	0	0
12	OPR	0	8
13	JMC	0	17
14	LIT	0	0
15	STO	1	5
16	JMP	0	26
17	LOD	1	3
18	LIT	0	5
19	OPR	0	8
20	JMC	0	24
21	LIT	0	1
22	STO	1	5
23	JMP	0	26
24	LIT	0	0
25	STO	1	5
26	LOD	1	5
27	STO	1	3
28	RET	0	0
29	LIT	0	5
30	CAL	0	7
31	INT	0	-1
32	LOD	0	3
33	STO	0	4
34	RET	0	0
\end{lstlisting}

\chapter{Závěr}

\end{document}
